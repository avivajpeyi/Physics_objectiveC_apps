%%%%%%%%%%%%%%%%%%%%%%%%%%%%%%%%%%%%%%%%%
% University/School Laboratory Report
% LaTeX Template
% Version 3.1 (25/3/14)
%
% This template has been downloaded from:
% http://www.LaTeXTemplates.com
%
% Original author:
% Linux and Unix Users Group at Virginia Tech Wiki 
% (https://vtluug.org/wiki/Example_LaTeX_chem_lab_report)
%
% License:
% CC BY-NC-SA 3.0 (http://creativecommons.org/licenses/by-nc-sa/3.0/)
%
%%%%%%%%%%%%%%%%%%%%%%%%%%%%%%%%%%%%%%%%%

%----------------------------------------------------------------------------------------
%	PACKAGES AND DOCUMENT CONFIGURATIONS
%----------------------------------------------------------------------------------------

\documentclass{article}

\usepackage[version=3]{mhchem} % Package for chemical equation typesetting
\usepackage{siunitx} % Provides the \SI{}{} and \si{} command for typesetting SI units
\usepackage{graphicx} % Required for the inclusion of images
\usepackage{natbib} % Required to change bibliography style to APA
\usepackage{amsmath} % Required for some math elements 
\usepackage{gensymb}

\setlength\parindent{0pt} % Removes all indentation from paragraphs

\renewcommand{\labelenumi}{\alph{enumi}.} % Make numbering in the enumerate environment by letter rather than number (e.g. section 6)

%\usepackage{times} % Uncomment to use the Times New Roman font


%Code packages

\usepackage[utf8]{inputenc}
\usepackage{listings}
\usepackage{color}
% Code packages end


%----------------------------------------------------------------------------------------
%	DOCUMENT INFORMATION
%----------------------------------------------------------------------------------------

\title{Euler Wobble\\ PHY 230} % Title

\author{Avi \textsc{Vajpeyi}} % Author name

\date{\today} % Date for the report

\begin{document}

\maketitle % Insert the title, author and date



% If you wish to include an abstract, uncomment the lines below
% \begin{abstract}
% Abstract text
% \end{abstract}

%----------------------------------------------------------------------------------------
%	SECTION 1
%----------------------------------------------------------------------------------------

\section{Difficulties with the Program}

The same peculiar issue with the animation appeared again this week, where the values were being updated, but the drawing was not appearing. I thought that I had avoided this as I started by coping the spring app and working on that project by changing the code. However, some point while writing the code, I must have changed something to spoil the animation. I fixed this by quiting Xcode, re copying the spring app and copying my code into that. This fixed the problem.

I also had small typos, which were hard to debug. One was where I used a vector instead of a unit vector, which resulted in large values that give erroneous results.

\section {Euler Wobble}

Working on the theory of this app was very interesting. It helped me understand how we can describe the formation of shadows due to light, and how changing perspective changes the size of an object. It was really interesting to study this, and code this. It makes me wonder if the first initial 3D video games incorporated this, and if they are still using this.   

The app has sliders to change the perspective, the speed of rotation, and the lengths of the sides of the cuboids. We also have an option of drawing the cuboid with only its frames, and with faces. 

 
 
%----------------------------------------------------------------------------------------


\end{document}